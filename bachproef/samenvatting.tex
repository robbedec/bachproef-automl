%%=============================================================================
%% Samenvatting
%%=============================================================================

% TODO: De "abstract" of samenvatting is een kernachtige (~ 1 blz. voor een
% thesis) synthese van het document.
%
% Deze aspecten moeten zeker aan bod komen:
% - Context: waarom is dit werk belangrijk?
% - Nood: waarom moest dit onderzocht worden?
% - Taak: wat heb je precies gedaan?
% - Object: wat staat in dit document geschreven?
% - Resultaat: wat was het resultaat?
% - Conclusie: wat is/zijn de belangrijkste conclusie(s)?
% - Perspectief: blijven er nog vragen open die in de toekomst nog kunnen
%    onderzocht worden? Wat is een mogelijk vervolg voor jouw onderzoek?
%
% LET OP! Een samenvatting is GEEN voorwoord!

%%---------- Nederlandse samenvatting -----------------------------------------
%
% TODO: Als je je bachelorproef in het Engels schrijft, moet je eerst een
% Nederlandse samenvatting invoegen. Haal daarvoor onderstaande code uit
% commentaar.
% Wie zijn bachelorproef in het Nederlands schrijft, kan dit negeren, de inhoud
% wordt niet in het document ingevoegd.

\IfLanguageName{english}{%
\selectlanguage{dutch}
\chapter*{Samenvatting}
\lipsum[1-4]
\selectlanguage{english}
}{}

%%---------- Samenvatting -----------------------------------------------------
% De samenvatting in de hoofdtaal van het document

\chapter*{\IfLanguageName{dutch}{Samenvatting}{Abstract}}

De resultaten van dit onderzoek kunnen gebruikt worden om een keuze te maken tussen een open source of cloud platform waar geautomatiseerde \textit{machine learning} gebruikt kan worden. Alsook in welke situaties het bruikbaar is. Het kan een hulp zijn voor bedrijven waar geen \textit{data scientist} of ML-ingenieurs aanwezig zijn. Of waar deze gewoonweg niet genoeg tijd hebben voor de verschillende projecten. Deze technologie tracht \textit{machine learning} toegankelijker te maken door een manier te bieden om vaak voorkomende problemen te automatiseren. Dit werd onderzocht door met AutoKeras en Google Cloud AutoML elk een prototype op te zetten dat voor een simpel maar realistisch classificatieprobleem de categorie van een afbeelding kan voorspellen. Er werden modellen getraind die katten van honden kunnen onderscheiden. Dit document beschrijft een studie naar de achterliggende gebruikte technieken, het verloop en de resultaten van beide prototypes. Er werd gevonden dat de alternatieven elk hun plaats hebben in verschillende fasen van een project. Google Cloud AutoML levert een productie waardig model terwijl AutoKeras kan dienen als hulpmiddel voor een \textit{data scientist} of productie waardig kan zijn mits een extensieve voorbereiding van de data. De evolutie van de platformen zelf betekent enkel goed nieuws voor de toekomst. Mogelijks kan er nog onderzocht worden hoe het opschonen van de data geautomatiseerd kan worden. Dit is een grote stap binnen geautomatiseerde \textit{machine learning} aangezien het een belangrijke factor is om \textit{edge cases} te herkennen.