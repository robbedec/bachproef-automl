%%=============================================================================
%% Conclusie
%%=============================================================================

\chapter{Conclusie}
\label{ch:conclusie}

% TODO: Trek een duidelijke conclusie, in de vorm van een antwoord op de
% onderzoeksvra(a)g(en). Wat was jouw bijdrage aan het onderzoeksdomein en
% hoe biedt dit meerwaarde aan het vakgebied/doelgroep? 
% Reflecteer kritisch over het resultaat. In Engelse teksten wordt deze sectie
% ``Discussion'' genoemd. Had je deze uitkomst verwacht? Zijn er zaken die nog
% niet duidelijk zijn?
% Heeft het onderzoek geleid tot nieuwe vragen die uitnodigen tot verder 
%onderzoek?

In eerste instantie werd er (indirect) onderzocht als zo'n ingewikkeld proces wel geautomatiseerd kan worden. De resultaten tonen aan dat AutoML wel degelijk een plaats verdient in de wereld van \textit{machine learning}. De hoofdvraag blijft natuurlijk de bruikbaarheid voor bedrijven en daar zijn toch enkele opmerkingen over. Zo zal er zeker een afweging plaatsvinden waarbij kwaliteit tegenover kost wordt gezet. AutoML is niet goedkoop, zeker op een cloud platform dat snel drie à vierduizend euro per maand kan kosten om operationeel te blijven. Voor AutoKeras zijn de kosten op het eerste zicht beperkt tot verbruikte elektriciteit en \textit{deployment}. Men moet daarbij rekening houden dat elke stap zelf geprogrammeerd moet worden en er toch enige kennis voor nodig is. Het uiteindelijk resultaat wordt dan deels bepaald door de \textit{data preprocessing} die ook handmatig moet gebeuren, deze stap is een belangrijke \textit{trigger} om hoge scores te behalen zoals bij Google Cloud AutoML.

Beide systemen komen de verwachtingen na maar moeten op de juiste plaats ingezet worden. Zo is Google Cloud AutoML een volwaardig \textit{drop in replacement} in bestaande applicaties. Het proces kan niet eenvoudiger zijn en de verschillende manieren om het te integreren zorgen ervoor dat het in meeste situaties past. AutoKeras, in zijn huidige staat, is niet verfijnd genoeg om productie waardig te zijn. De extra moeite om de eerste stappen van het procesmodel te verbeteren kan evengoed verwisseld worden met een ML-ingenieur die het volledige proces uitvoert. Die niche kennis blijft noodzakelijk om te slagen. Anderzijds blijkt het wel een goede \textit{tool} te zijn in de gereedschapskist van ML-ingenieurs enzovoort. Stel een situatie voor waarbij zo snel mogelijk een MVP\footnote{Minimum viable product} voorgesteld moet worden. Zonder al te veel moeite kan er met AutoKeras een basis gelegd worden, ook kan het een andere inkijk over het probleem geven die misschien nog niet overwogen was. 

Let wel op, Google Cloud AutoML heeft ook zijn nadelen. Naast het stevig kostenplaatje is er ook nog de \textit{vendor lock in} op de cloud. Zo werd er voor een kleine toepassing al gebruik gemaakt van \textit{storage servers}, \textit{vision API} en een \textit{deployment platform}. Als klant is het niet de bedoeling dat een applicatie volledig afhankelijk is van één platform. Het neemt deels de vrijheid af, zo is het niet mogelijk om de structuur van het model te zien. 

De toekomst van geautomatiseerde \textit{machine learning} ziet er alvast goed uit. De verbeteringen tussen versies van AutoKeras vallen op en ook steeds meer cloud platformen bieden een gelijkaardige service aan. Er is een echte \textit{push} aan de gang, van de \textit{community} en de bedrijven, om de toepassingen toegankelijker te maken. Verder onderzoek over dit onderwerp zou zich kunnen richten op individuele stappen van het procesmodel, bijvoorbeeld de \textit{data preprocessing}. De automatisatie ervan is niet vanzelfsprekend omdat dit voor elke dataset anders is.