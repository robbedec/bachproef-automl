%%=============================================================================
%% Voorwoord
%%=============================================================================

\chapter*{\IfLanguageName{dutch}{Woord vooraf}{Preface}}
\label{ch:voorwoord}

%% TODO:
%% Het voorwoord is het enige deel van de bachelorproef waar je vanuit je
%% eigen standpunt (``ik-vorm'') mag schrijven. Je kan hier bv. motiveren
%% waarom jij het onderwerp wil bespreken.
%% Vergeet ook niet te bedanken wie je geholpen/gesteund/... heeft

De keuze van mijn onderwerp heb ik snel kunnen maken. In mijn vrije tijd experimenteerde ik al eens graag met Tensorflow en later tijdens de opleiding keerde dit terug. In de lessen is het veel uitgebreider aan bod gekomen en werd mijn interesse voor wat er achter de schermen gebeurt groter. De meeste onderwerpen die iets de maken hebben met \textit{machine learning} zijn uitdagend. De abstracte theorieën, gigantische hoeveelheid data ... Bij mij was dit niet anders, maar ik hoop om deze nieuwe kennis later te kunnen gebruiken in mijn carrière.

Deze bachelorproef werd geschreven in het kader van het voltooien van de opleiding Toegepaste Informatica en betekent ook het einde van een periode. Hierbij wil ik graag een aantal mensen bedanken die mij geholpen hebben bij het uitvoeren van dit onderzoek.

Eest en vooral zou ik graag mijn promotor, ir. Johan Decorte, bedanken. Tijdens een periode waar alles digitaal moest verlopen kon ik steeds bij hem terecht met vragen over de bachelorproef. Zijn feedback over de inhoud heeft mij vaak geholpen om alles ineen te steken.

Ook wil ik mijn co-promotor, dr. Kenny Helsens, bedanken voor het sturen van de inhoud. Tijdens de gesprekken in de kantoren van In The Pocket konden we samen brainstormen naar aanvullingen van het onderwerp. Daarnaast pushte hij me om het breder te bekijken en zo zijn er een aantal zaken naar boven gekomen waar ikzelf nooit aan gedacht zou hebben, mocht het niet door onze gesprekken zijn.

Om af te sluiten wil ik ook mijn ouders bedanken die mij van begin tot einde van de opleiding ondersteund hebben.